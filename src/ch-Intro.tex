\renewcommand*{\PicDir}{pic/ch-Intro}

\chapter{Introduction}
\vspace{2em}

\lipsum[1]

A book~\cite{BOOK:Gray} is cited.
\lipsum[2]

An online article~\cite{tpc-c} is cited.
\lipsum[3]

The compound word ``ingredient\zz{}insensitive'', where the hyphen is generated through command \CMD{\zz{}} will be hyphenated for individual words rather than the compound word as a whole. 
In contrast, ``nutrition-oriented'' with normal hyphen will be hyphenated as a whole compound word, which is unlikely to be recognized by \LaTeX{} and therefore no hyphenation will be carried out unless you provide customized hyphenation for it. 

\begin{figure}[!t]
  \centering
  \includegraphics[width=.6\linewidth]{\Pic{pdf}{food}}
  \vspace{\BeforeCaptionVSpace}
  \caption{A collection of food.}
  \label{intro:fig:food}
\end{figure}

\section{Overview}

\autoref{intro:fig:food} demonstrates a collection of food. 
\lipsum[6-10]

\section{Thesis Synopsis}

The rest of this thesis is organized as follows. 
\lipsum[11]
